\part{Production Optimisation}
%%%%%%%%%%%%%%%%%%%%%%%%%%%%%%%%%%%%%%%%%%%%%%%%%%%%%%%%%%%%%%%%%%%%%%%%%
%   Production Optinmisation                                            %
%%%%%%%%%%%%%%%%%%%%%%%%%%%%%%%%%%%%%%%%%%%%%%%%%%%%%%%%%%%%%%%%%%%%%%%%%

\section{Motivation and background}
Although a large portion of time spent in Surewash involves the continuous development and maintenance of their products, a critical function of the company is the actual manufacturing of the {\slshape ELITE} and {\slshape GO}, hereafter the {\slshape process}. A lot of the process has been outsourced. As an example, the computer for the {\slshape GO} is a {\slshape Microsoft Surface Pro}. This goes inside a hard plastic casing which is manufactured by a plastics manufacturer in China. The assembly of all the hardware is done by another company in Dublin.

The final stage of the process involves installing all relevant software, hereafter called commissioning. In the commissioning stage. The computers come pre-installed with {\slshape Windows 10}, but there are configuration steps that are required to be completed before the product is shipped to the end user. As a summary of these steps are as follows:
\begin{enumerate}
    \item Turn on the computer and go through the steps of the {\slshape Windows 10} setup wizard.
    \item Change some of the default {\slshape Windows 10} settings (e.g. turn off all the notification types, change some battery settings, change the wallpaper, etc)
    \item Allow {\slshape Windows Update} to complete all relevant updates (this can take up to several hours sometimes)
    \item Connect a Surewash USB key, open the Surewash commissioning app, let {\slshape Windows 10} install the .NET framework (can take 10-20 minutes)
    \item Complete each of the steps specified in the commissioning app, but ignore some of the steps.
\end{enumerate}

The steps in the commissioning app require a lot of user input to implement correctly. In short, the commissioning process is long, laborious, and prone to human error. This leads problems both in quality control, and in wasted time for Surewash employees. While there is certainly scope to streamline the current process, it is my opinion that the entire process needs to be rethought.

\section{A new framework for commissioning}
    \subsection{Introduction}
    The concept of preparing a `golden image' of a device's hard drive, and then rolling that image out to other computers is a well-documented, and common practice amongst organisations (\cite{msftdeployment}). In short, one computer is set up to the desired end state, it is then booted into a special operating system, where the contents of one or more partitions in the hard drive/ SSD is copied, this copy is then transferred onto other devices, either manually, or by some automatic method, such as using a PXE server.

    There are various solutions that achieve this, both free and paid. Clonezilla is the solution that was chosen to achieve this \cite{clonezilla}. It was chosen because it is free (GPL License), and runs on Debian, which is also free.
    \subsection{The new process}
    The old steps of commissioning now need only happen on one computer, but with some caveats. There are a few steps that involve unique input, that is, the input of unique passwords and IDs. A fundamental limitation of Clonezilla, and indeed any other similar solution, is that it cannot perform these steps. Once these non-unique steps have been done, Clonezilla is used to take an image of the Computer's disk, and this is flashed onto other computers. The last unique steps such as passwords are then done on all of the computers being comissioned individually.

    \section{Conclusion}
    \subsection{Learning Outcomes}
        \subsubsection{Other}
            \paragraph{Licensing}
            I learned about the consequences of using GPL-licensed software, and what copyleft means, and why it might be problematic for a company like Surewash to integrate GPL-licensed software into its own, since it is all closed-source.